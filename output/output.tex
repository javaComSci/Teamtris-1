\documentclass[12pt]{article}
\usepackage{listings}
\usepackage{graphicx}
\usepackage{minted}
\usepackage[T1]{fontenc}
\newlength{\drop}

\begin{document}
  \begin{titlepage}
    \centering
    \drop=0.1\textheight
    \vspace*{7\baselineskip}
    \rule{\textwidth}{1.6pt}\vspace*{-\baselineskip}\vspace*{2pt}
    \rule{\textwidth}{0.4pt}\\[\baselineskip]
    {\LARGE TeamTris}\\[0.2\baselineskip]
    \rule{\textwidth}{0.4pt}\vspace*{-\baselineskip}\vspace{3.2pt}
    \rule{\textwidth}{1.6pt}\\[\baselineskip]
    \scshape
    TeamTris Code Documentation \\
    West Lafayette, IN \\
    March 5th 2020\par
    \vspace*{2\baselineskip}
    Created by \\[\baselineskip]
    {\Large Steven Dellamore \\ Indhu Ramanathan \\ Richard Hansen \\ Columbus Holt\par}
    {\itshape dellamoresteven@gmail.com\par}
    {\itshape TeamTris \\ CS407\par}
  \end{titlepage}
  
\tableofcontents
\newpage


\section{StartScreen}
\textbf{Author}: Steven Dellamore \\
\textbf{Description}: Startscreen will build the startscreen and create all the buttons needed for the user to get into a game with their friends. The mouseClicks and the keyboard imports all all forwarded to this class when \mintinline[fontsize=\footnotesize]{javascript}{gamestate == 0} \\



\subsection{constructor}
\textbf{Author}: Steven Dellamore 
\vspace*{1\baselineskip}
\begin{lstlisting}
constructor()
\end{lstlisting} 
\vspace*{1\baselineskip}
\textbf{Description}: The constructor gets called when making a startscreen object. It will init all the values and set up the socket listener for the server to send things too. Here are the init values of the class variables: 
\begin{minted}[fontsize=\footnotesize]{javascript}
this.TokenBoxText = ""; 
this.usernameBoxStroke = false; 
this.usernameText = "username"; 
this.usernameTextTouched = false; 
this.gameStateStartScreen = 0;
this.titleAnimation = [300, 500, 400, 700];
\end{minted}
 These varibles will be updated throughout the life of start screen. \mintinline[fontsize=\footnotesize]{javascript}{this.TokenBoxText} will init the token box to nothing, since the user has yet to do anyhting. the \mintinline[fontsize=\footnotesize]{javascript}{this.usernameBoxStroke} will be set to false so the program knows if the user as tried to sumbit. \mintinline[fontsize=\footnotesize]{javascript}{this.titleAnimation = [300, 500, 400, 700];} is the starting position of the title, and will fall every X frames. 


\textbf{\large{\\Parameters}}:\\
\textbf{void }: constructor takes no params\\\textbf{\large{\\Returns}}:\\\textbf{StartScreen }: An object of start class class

\subsection{draw}
\textbf{Author}: Steven Dellamore 
\vspace*{1\baselineskip}
\begin{lstlisting}
draw()
\end{lstlisting} 
\vspace*{1\baselineskip}
\textbf{Description}: This funcion will be ran at 60 frames a second and will call all the functions needed to draw the launch screen. The draw function will call the title functions, the highscore functions, and call the join and create button rendering/hitboxes with \mintinline[fontsize=\footnotesize]{javascript}{Buttonloop()}. Depending on what \mintinline[fontsize=\footnotesize]{javascript}{this.gameStateStartScreen} is evaluated to. 
\begin{minted}[fontsize=\footnotesize]{javascript}
switch (this.gameStateStartScreen) {
	case 0:
		this.drawUsernameBox(); 
		break;
	case 1:
		this.drawTokenBox();
		break;
}
\end{minted}
 


\textbf{\large{\\Parameters}}:\\
\textbf{void }: draw takes no arugments\\\textbf{\large{\\Returns}}:\\\textbf{void }: void

\subsection{animateTitle}
\textbf{Author}: Steven Dellamore 
\vspace*{1\baselineskip}
\begin{lstlisting}
animateTitle()
\end{lstlisting} 
\vspace*{1\baselineskip}
\textbf{Description}: Will check and add/subtract the locations of the T's falling when you go to the launch screen. 
\begin{minted}[fontsize=\footnotesize]{javascript}
if (this.titleAnimation[i] > 0) {
	this.titleAnimation[i] -= 10;
}
\end{minted}
 Once \mintinline[fontsize=\footnotesize]{javascript}{this.titleAnimation[i]}, where \mintinline[fontsize=\footnotesize]{javascript}{i} is between \mintinline[fontsize=\footnotesize]{javascript}{[0,4]}, is negative, the array index will no longer be decremented. 


\textbf{\large{\\Parameters}}:\\
\textbf{void }: animateTitle takes no arugments\\\textbf{\large{\\Returns}}:\\\textbf{void}

\subsection{drawUsernameBox}
\textbf{Author}: Steven Dellamore 
\vspace*{1\baselineskip}
\begin{lstlisting}
drawUsernameBox()
\end{lstlisting} 
\vspace*{1\baselineskip}
\textbf{Description}: This function will draw the white username box onto the screen displaying the \mintinline[fontsize=\footnotesize]{javascript}{this.usernameText} in the center. This function will also use \mintinline[fontsize=\footnotesize]{javascript}{this.usernameBoxStroke} to display the red outline around the username box. 


\textbf{\large{\\Parameters}}:\\
\textbf{void }: drawUsernameBox takes no arugments\\\textbf{\large{\\Returns}}:\\\textbf{void}

\subsection{drawTitle}
\textbf{Author}: Steven Dellamore 
\vspace*{1\baselineskip}
\begin{lstlisting}
drawTitle()
\end{lstlisting} 
\vspace*{1\baselineskip}
\textbf{Description}: This function will draw the title (Teamtris) onto the launch screen. Also, the function will be responable for displaying the current falling location of the two T's falling at the start of the screen. We make rects based on the current location of \mintinline[fontsize=\footnotesize]{javascript}{this.titleAnimation}. 
\begin{minted}[fontsize=\footnotesize]{javascript}
let yStart;
rect(-windowWidth / 4.3, (yStart = windowHeight / 2.6) - 
		this.titleAnimation[0], squareSize, squareSize) 

rect(-windowWidth / 4.3, (yStart - (spaceBetweenSquares)) - 
		this.titleAnimation[0], squareSize, squareSize) 

fill(255, 0, 0) // fill red

rect(-windowWidth / 4.3, yStart - (2 * spaceBetweenSquares) - 
		this.titleAnimation[1], squareSize, squareSize)

rect(-windowWidth / 4.3 - spaceBetweenSquares, 
		yStart - (2 * spaceBetweenSquares) - this.titleAnimation[1], 
				squareSize, squareSize)

rect(-windowWidth / 4.3 + spaceBetweenSquares, 
		yStart - (2 * spaceBetweenSquares) - this.titleAnimation[1], 
				squareSize, squareSize)
\end{minted}
 The important thing to note is to see the y val of the rect is being changed by 10 every frame in \mintinline[fontsize=\footnotesize]{javascript}{function animateTitle()}. 


\textbf{\large{\\Parameters}}:\\
\textbf{void }: drawTitle takes no arugments\\\textbf{\large{\\Returns}}:\\\textbf{void}

\subsection{drawTokenBox}
\textbf{Author}: Steven Dellamore 
\vspace*{1\baselineskip}
\begin{lstlisting}
drawTokenBox()
\end{lstlisting} 
\vspace*{1\baselineskip}
\textbf{Description}: This function will draw the token box once the user clicks "join game". It will display the token box and the accept button. Unlike other buttons, all mouse clicks are handled. 


\textbf{\large{\\Parameters}}:\\
\textbf{void }: drawTokenBox takes no arugments\\\textbf{\large{\\Returns}}:\\\textbf{void}

\subsection{mouseClickedStart}
\textbf{Author}: Steven Dellamore 
\vspace*{1\baselineskip}
\begin{lstlisting}
mouseClickedStart()
\end{lstlisting} 
\vspace*{1\baselineskip}
\textbf{Description}: This function is being called whenever \mintinline[fontsize=\footnotesize]{javascript}{gamestate = 0} AND the user clicks their mouse. First, we will check what \mintinline[fontsize=\footnotesize]{javascript}{this.gameStateStartScreen} is. If its \mintinline[fontsize=\footnotesize]{javascript}{0}, we will check the \mintinline[fontsize=\footnotesize]{javascript}{function ClickedLoop()} to see if the user is clicking on the join game, create game, or highscore score buttons. If the user clicks on a the create game button with a valid username we are going to send them into the lobbyscreen. 
\begin{minted}[fontsize=\footnotesize]{javascript}
// Creating my lobbyscreen object
mLobbyScreen = new LobbyScreen(
	new Player(
		this.usernameText, Math.floor(Math.random() * 100), true));

gameState = 1; // Switch to lobby screen
\end{minted}
 We need to create a new Player, and set their ownership value to 0. We see its constructor defined here: 
\begin{minted}[fontsize=\footnotesize]{javascript}
constructor(username, id, owner){
    this.username = username;
    this.id = id;
    this.owner = owner;
    this.playerNum;
}
\end{minted}
 We then pass this object into the lobbyscreen and switch the \mintinline[fontsize=\footnotesize]{javascript}{gameState = 1} to move the user to the next screen. 


\textbf{\large{\\Parameters}}:\\
\textbf{void }: mouseClickedStart takes no arugments\\\textbf{\large{\\Returns}}:\\\textbf{void}

\subsection{drawHighScoreButtonCheckMouse}
\textbf{Author}: Steven Dellamore 
\vspace*{1\baselineskip}
\begin{lstlisting}
drawHighScoreButtonCheckMouse()
\end{lstlisting} 
\vspace*{1\baselineskip}
\textbf{Description}: This function is being called whenever the user clicks with gamestate of the \mintinline[fontsize=\footnotesize]{javascript}{this.gameStateStartScreen == 0;}. This function checks if the mouse is over the highscore button and returns \mintinline[fontsize=\footnotesize]{javascript}{true} if it is, \mintinline[fontsize=\footnotesize]{javascript}{false} if its not. 


\textbf{\large{\\Parameters}}:\\
\textbf{void }: drawHighScoreButtonCheckMouse takes no arugments\\\textbf{\large{\\Returns}}:\\\textbf{bool }: \\true => If mouse is over score button \\false => If mouse is not over score button

\subsection{drawHighScoreButton}
\textbf{Author}: Steven Dellamore 
\vspace*{1\baselineskip}
\begin{lstlisting}
drawHighScoreButton()
\end{lstlisting} 
\vspace*{1\baselineskip}
\textbf{Description}: This function will draw the three bars in the bottom left of the screen. It will first check what \mintinline[fontsize=\footnotesize]{javascript}{this.drawHighScoreButtonCheckMouse()} and set accordingly: 
\begin{minted}[fontsize=\footnotesize]{javascript}
let fillHighScore = "white"; // default value is white
/* Checks if the mouse is over the highscore */
if (this.drawHighScoreButtonCheckMouse()) { 
	/* if the mouse is over, it will change the boxes to green */
	fillHighScore = "rgb(0,255,0)"; 
}
\end{minted}
 If \mintinline[fontsize=\footnotesize]{javascript}{this.drawHighScoreButtonCheckMouse()} returns true, then we set \mintinline[fontsize=\footnotesize]{javascript}{fillHighScore} to \mintinline[fontsize=\footnotesize]{javascript}{"green"}, otherwise keep it \mintinline[fontsize=\footnotesize]{javascript}{"white"}. 


\textbf{\large{\\Parameters}}:\\
\textbf{void }: takes no arugments\\\textbf{\large{\\Returns}}:\\\textbf{void }: no return

\subsection{keyPressedStart}
\textbf{Author}: Steven Dellamore 
\vspace*{1\baselineskip}
\begin{lstlisting}
keyPressedStart()
\end{lstlisting} 
\vspace*{1\baselineskip}
\textbf{Description}: Called whenever the \mintinline[fontsize=\footnotesize]{javascript}{General::function keyPressed()} function routes the signal to this function. a.k.a whenver \mintinline[fontsize=\footnotesize]{javascript}{gameState == 0}. This function first checks the \mintinline[fontsize=\footnotesize]{javascript}{this.gameStateStartScreen} like so: 
\begin{minted}[fontsize=\footnotesize]{javascript}
switch(this.gameStateStartScreen) {
    case 0:
        // username box active
        ...
    case 1:
        // token box active
        ...
}
\end{minted}
 From here, we can figure out where the user is trying to type and add the types characters accoridngly. 


\textbf{\large{\\Parameters}}:\\
\textbf{void }: keyPressedStart takes no arugments\\\textbf{\large{\\Returns}}:\\\textbf{void}

\section{Player}
\textbf{Author}: Steven Dellamore, Richard Hansen \\
\textbf{Description}: Every user will have their own object of the Player class. This is going to be passed around to other people in the lobby. This class will tell the game screen who is who and will help identify moves. \\



\subsection{constructor}
\textbf{Author}: Steven Dellamore 
\vspace*{1\baselineskip}
\begin{lstlisting}
constructor(username, id, owner)
\end{lstlisting} 
\vspace*{1\baselineskip}
\textbf{Description}: The constructor takes in three things, a name, id and a owner flag. It will then create an object of \mintinline[fontsize=\footnotesize]{javascript}{Player} and init all class varibles. This Class is used throughout all stages of the program. 


\textbf{\large{\\Parameters}}:\\
\textbf{String username }: username of the new Player\\
\textbf{int id }: id, \mintinline[fontsize=\footnotesize]{javascript}{[0,4]}, of the new player.\\
\textbf{boolean owner }: \mintinline[fontsize=\footnotesize]{javascript}{true} or \mintinline[fontsize=\footnotesize]{javascript}{false} if they are owner\\\textbf{\large{\\Returns}}:\\\textbf{Player }: An object of Player class

\subsection{setPlayerNum}
\textbf{Author}: Steven Dellamore 
\vspace*{1\baselineskip}
\begin{lstlisting}
setPlayerNum(num)
\end{lstlisting} 
\vspace*{1\baselineskip}
\textbf{Description}: Will set \mintinline[fontsize=\footnotesize]{javascript}{this.playerNum} equal to \mintinline[fontsize=\footnotesize]{javascript}{num}. This is just a helper function. 


\textbf{\large{\\Parameters}}:\\
\textbf{int num }: sets the \mintinline[fontsize=\footnotesize]{javascript}{this.playerNum = num}\\\textbf{\large{\\Returns}}:\\\textbf{void }: returns nothing

\section{Team}
\textbf{Author}: Steven Dellamore, Richard Hansen \\
\textbf{Description}: The team class will contain all the other players that are in your game, the team name and the token for your lobby. Once new players come addPlayer will be called to push a newplayer onto the playersInTeam array. \\



\subsection{constructor}
\textbf{Author}: Steven Dellamore 
\vspace*{1\baselineskip}
\begin{lstlisting}
constructor()
\end{lstlisting} 
\vspace*{1\baselineskip}
\textbf{Description}: The constructor gets called anytime someone joins or create a game. 


\textbf{\large{\\Parameters}}:\\
\textbf{void}: no parameters\\\textbf{\large{\\Returns}}:\\\textbf{Team }: A object of the class

\subsection{addPlayer}
\textbf{Author}: Steven Dellamore 
\vspace*{1\baselineskip}
\begin{lstlisting}
addPlayer(player)
\end{lstlisting} 
\vspace*{1\baselineskip}
\textbf{Description}: The add player function gets called whenever a bot or a real player joins your lobby. This function will also be called to populate the lobby when you join. 


\textbf{\large{\\Parameters}}:\\
\textbf{Player player}: This parameter is the new player/bot that is joining your team.\\\textbf{\large{\\Returns}}:\\\textbf{void }: no return

\section{General}
\textbf{Author}: Steven Dellamore, Richard Hansen \\
\textbf{Description}: This is an abstract class that will hold mouseClicked and keyPressed p5 functions. \\



\subsection{mouseClicked}
\textbf{Author}: Steven Dellamore, Richard Hansen 
\vspace*{1\baselineskip}
\begin{lstlisting}
mouseClicked()
\end{lstlisting} 
\vspace*{1\baselineskip}
\textbf{Description}: Will be called whenever the user clicks on anywhere on the screen. Once called, it will go straight into a switch to decide where to route to based on the gameState 
\begin{minted}[fontsize=\footnotesize]{javascript}
switch (gameState) {
    case 0:
        // start screens mouseClicked
        mStartScreen.mouseClickedStart(); 
        break;
    case 1:
        // lobby screens mouseClicked
        mLobbyScreen.mouseClickedLobby();
        break;
    case 2:
        break;
    case 3:
        break;
}
\end{minted}
 The varibles \mintinline[fontsize=\footnotesize]{javascript}{gameState}, \mintinline[fontsize=\footnotesize]{javascript}{mStartScreen}, \mintinline[fontsize=\footnotesize]{javascript}{mLobbyScreen} are all defined in sketch.js 


\textbf{\large{\\Parameters}}:\\
\textbf{void }: takes no parameters\\\textbf{\large{\\Returns}}:\\\textbf{void }: returns nothing

\subsection{keyPressed}
\textbf{Author}: Steven Dellamore, Richard Hansen 
\vspace*{1\baselineskip}
\begin{lstlisting}
mouseClicked()
\end{lstlisting} 
\vspace*{1\baselineskip}
\textbf{Description}: Will be called whenever the presses a key. Once called, it will go straight into a switch to decide where to route to based on the gameState 
\begin{minted}[fontsize=\footnotesize]{javascript}
switch (gameState) {
  case 0:
    mStartScreen.keyPressedStart();
    break;
  case 1:
    mLobbyScreen.keyPressedLobby();
    break;
  case 2:
    mGameScreen.keyPressedGame();
    break;
  case 3:
    // mScoreScreen.keyPressedScore();
    break;
}
\end{minted}
 The varibles \mintinline[fontsize=\footnotesize]{javascript}{gameState}, \mintinline[fontsize=\footnotesize]{javascript}{mStartScreen}, \mintinline[fontsize=\footnotesize]{javascript}{mLobbyScreen} are all defined in sketch.js 


\textbf{\large{\\Parameters}}:\\
\textbf{void }: takes no parameters\\\textbf{\large{\\Returns}}:\\\textbf{void }: returns nothing

\section{Block}
\textbf{Author}: ??? \\
\textbf{Description}: TODO \\



\section{Board}
\textbf{Author}: ??? \\
\textbf{Description}: Board class to include the information that the board will need to have for a bot \\



\subsection{board}
\textbf{Author}: ??? 
\vspace*{1\baselineskip}
\begin{lstlisting}
public int[,] board()
\end{lstlisting} 
\vspace*{1\baselineskip}
\textbf{Description}: TODO 


\textbf{\large{\\Parameters}}:\\
\textbf{type name }: what it does\\\textbf{\large{\\Returns}}:\\\textbf{type }: what it does

\subsection{numFilledFloor}
\textbf{Author}: ??? 
\vspace*{1\baselineskip}
\begin{lstlisting}
public int numFilledFloor()
\end{lstlisting} 
\vspace*{1\baselineskip}
\textbf{Description}: has the number of heights that are not 0 - meaning that they have already been filled 


\textbf{\large{\\Parameters}}:\\
\textbf{type name }: what it does\\\textbf{\large{\\Returns}}:\\\textbf{type }: what it does

\section{Play}
\textbf{Author}: ??? \\
\textbf{Description}: TODO \\



\subsection{Play}
\textbf{Author}: ??? 
\vspace*{1\baselineskip}
\begin{lstlisting}
public Play()
\end{lstlisting} 
\vspace*{1\baselineskip}
\textbf{Description}: TODO 



\section{Player}
\textbf{Author}: ??? \\
\textbf{Description}: TODO \\



\section{Prints}
\textbf{Author}: ??? \\
\textbf{Description}: TODO \\



\section{Program}
\textbf{Author}: ??? \\
\textbf{Description}: TODO \\



\section{SingleBot}
\textbf{Author}: JavaComSci \\
\textbf{Description}: Single bot extends the abstract bot class defined here: 
\begin{minted}[fontsize=\footnotesize]{csharp}
public abstract class Bot {
    public abstract List<Tuple<int, int>> GetMove(
        Board board, 
        List<Block> blocks, 
        bool allRotations = false
    );
}
\end{minted}
 The SingleBot class will be made if the player requires only one bot in their game. \\



\subsection{SingleBot}
\textbf{Author}: JavaComSci 
\vspace*{1\baselineskip}
\begin{lstlisting}
SingleBot()
\end{lstlisting} 
\vspace*{1\baselineskip}
\textbf{Description}: Creates a new board for the bot. 


\textbf{\large{\\Parameters}}:\\
\textbf{void }: SingleBot takes no params\\\textbf{\large{\\Returns}}:\\\textbf{SingleBot }: An object of single bot class

\subsection{getFit}
\textbf{Author}: JavaComSci 
\vspace*{1\baselineskip}
\begin{lstlisting}
List<...> getFit(Board board, Block block, int rotation)
\end{lstlisting} 
\vspace*{1\baselineskip}
\textbf{Description}: need desc here TODO 


\textbf{\large{\\Parameters}}:\\
\textbf{Board board }: contains the the board that we want to make the move on\\
\textbf{Block block }: contains the block that we want to fit\\
\textbf{int rotation }: which roation we are trying to fit for\\\textbf{\large{\\Returns}}:\\\textbf{List<...> compatiblePieces }: information about the pieces that are compatible on the board

\subsection{GetMove}
\textbf{Author}: JavaComSci 
\vspace*{1\baselineskip}
\begin{lstlisting}
public override List<...> GetMove(Board, List<Block>, bool)
\end{lstlisting} 
\vspace*{1\baselineskip}
\textbf{Description}: need desc here TODO 


\textbf{\large{\\Parameters}}:\\
\textbf{int[][] board }: current enviornment\\
\textbf{List<Block> blocks }: contains the list of all the blocks to try to fit in this location\\\textbf{\large{\\Returns}}:\\\textbf{List<...> bestPiecePlacementOfCurrentBlock }: contains the list of the indicies of where the piece would be on the board

\section{DoubleBot}
\textbf{Author}: JavaComSci \\
\textbf{Description}: DoubleBot bot extends the abstract bot class defined here: 
\begin{minted}[fontsize=\footnotesize]{csharp}
public abstract class Bot {
    public abstract List<Tuple<int, int>> GetMove(
        Board board, 
        List<Block> blocks, 
        bool allRotations = false
    );
}
\end{minted}
 The DoubleBot class will be made if the player requires two bots in their game. \\



\section{TripleBot}
\textbf{Author}: JavaComSci \\
\textbf{Description}: TripleBot extends the abstract bot class defined here: 
\begin{minted}[fontsize=\footnotesize]{csharp}
public abstract class Bot {
    public abstract List<Tuple<int, int>> GetMove(
        Board board, 
        List<Block> blocks, 
        bool allRotations = false
    );
}
\end{minted}
 The TripleBot class will be made if the player requires three bots in their game. \\



\section{BotManager}
\textbf{Author}: JavaComSci \\
\textbf{Description}: TODO \\



\section{FrontendTests}
\textbf{Author}: Steven Dellamore, Richard Hansen \\
\textbf{Description}: This is the testing doc for all the frontend tests. We decided to not go with a framework because we didnt think we needed everything the framework gives us. This framework uses the idea of dependency injection. We mock out all the p5 variables like so: 
\begin{minted}[fontsize=\footnotesize]{javascript}
global.mouseY = 30;
global.LEFT_ARROW = 37;
global.RIGHT_ARROW = 39;
global.DOWN_ARROW = 40;
global.createCanvas = function (x,y) { }
global.push = function () { }
global.pop = function () { }
global.translate = function () { }
... // Keeps going
\end{minted}
 This allows us to control all aspects of the test and really unit test every line of code in our functions. More over, we are able to mock out other classes that are being used by the class we are trying to test like so: 
\begin{minted}[fontsize=\footnotesize]{javascript}
global.buttonList = button[0];
global.Buttons = button[1];
global.Buttonloop = button[2];
global.ClickedLoop = button[3];
global.FindButtonbyID = button[4];
\end{minted}
 Once again, we can really drill down to the functions and have a really good understanding of what its doing and its return values. \\



\subsection{CheckSame}
\textbf{Author}: Steven Dellamore, Richard Hansen 
\vspace*{1\baselineskip}
\begin{lstlisting}
CheckSame( string, string, string, boolean = false )
\end{lstlisting} 
\vspace*{1\baselineskip}
\textbf{Description}: Checks to see if the \mintinline[fontsize=\footnotesize]{javascript}{given} and \mintinline[fontsize=\footnotesize]{javascript}{expected} strings are the same. If they are not this function will return false and print what the expected was. 
\begin{minted}[fontsize=\footnotesize]{javascript}
console.log(
    red, numTests++ + ". " + name + " failed should have been " + 
                                    expect + " but was " + given);
\end{minted}
 If its true it will print a success message. 


\textbf{\large{\\Parameters}}:\\
\textbf{string given }: real output\\
\textbf{string expect }: expected output\\
\textbf{string name }: name of test\\
\textbf{boolean debug }: \mintinline[fontsize=\footnotesize]{javascript}{true} if you want debug statments printed\\\textbf{\large{\\Returns}}:\\\textbf{boolean }: true if given and expected match, false otherwise

\subsection{testDefaultUsername}
\textbf{Author}: Steven Dellamore 
\vspace*{1\baselineskip}
\begin{lstlisting}
async function testDefaultUsername()
\end{lstlisting} 
\vspace*{1\baselineskip}
\textbf{Description}: Checks to see if the default \mintinline[fontsize=\footnotesize]{javascript}{mStartScreen.usernameText} is \mintinline[fontsize=\footnotesize]{javascript}{"username"}. 



\subsection{testDefaultTokenValue}
\textbf{Author}: Steven Dellamore 
\vspace*{1\baselineskip}
\begin{lstlisting}
async function testDefaultTokenValue()
\end{lstlisting} 
\vspace*{1\baselineskip}
\textbf{Description}: Checks to see if the default \mintinline[fontsize=\footnotesize]{javascript}{mStartScreen.TokenBoxText} is \mintinline[fontsize=\footnotesize]{javascript}{""}. 



\subsection{testCheckInitStartScreenValues}
\textbf{Author}: Steven Dellamore 
\vspace*{1\baselineskip}
\begin{lstlisting}
async function testCheckInitStartScreenValues()
\end{lstlisting} 
\vspace*{1\baselineskip}
\textbf{Description}: Checks to see if all the other init startscreen values are correct. 
\begin{minted}[fontsize=\footnotesize]{javascript}
// check usernameTextTouched is false
CheckSame(
    mStartScreen.usernameTextTouched,false,
        "checkInitStartScreenValues.usernameTextTouched");

// check titleAnimation [0-4] is set to the correct values
CheckSame(
    mStartScreen.titleAnimation[0],300,
        "checkInitStartScreenValues.titleAnimation[0]");
... // other indexs of titleAnimation

// Check the stroke of the box is set to false
CheckSame(
    mStartScreen.usernameBoxStroke,false,
        "checkInitStartScreenValues.usernameBoxStroke");
\end{minted}
 



\subsection{testCheckTitlePosAfterTwoDraw}
\textbf{Author}: Steven Dellamore 
\vspace*{1\baselineskip}
\begin{lstlisting}
async function testCheckTitlePosAfterTwoDraw()
\end{lstlisting} 
\vspace*{1\baselineskip}
\textbf{Description}: Run \mintinline[fontsize=\footnotesize]{javascript}{mStartScreen.draw()} twice and check that the title pos values have been updated correctly. 



\subsection{testChangeUserUsername}
\textbf{Author}: Steven Dellamore 
\vspace*{1\baselineskip}
\begin{lstlisting}
async function testChangeUserUsername()
\end{lstlisting} 
\vspace*{1\baselineskip}
\textbf{Description}: Will set the \mintinline[fontsize=\footnotesize]{javascript}{keyCode} equal to \mintinline[fontsize=\footnotesize]{javascript}{65} and \mintinline[fontsize=\footnotesize]{javascript}{66} and call the \mintinline[fontsize=\footnotesize]{javascript}{keyPressedStart()} function. Which tells the start screen that a key has been presesed. We then check if \mintinline[fontsize=\footnotesize]{javascript}{mStartScreen.usernameText} was changed to \mintinline[fontsize=\footnotesize]{javascript}{"A"} and \mintinline[fontsize=\footnotesize]{javascript}{"AB"}. 



\subsection{testChangeMaxUsername}
\textbf{Author}: Steven Dellamore 
\vspace*{1\baselineskip}
\begin{lstlisting}
async function testChangeMaxUsername()
\end{lstlisting} 
\vspace*{1\baselineskip}
\textbf{Description}: Will call the \mintinline[fontsize=\footnotesize]{javascript}{keyPressedStart()} function with letters ABCDEFGHIJKLMNOPQRS and check to ensure that the \mintinline[fontsize=\footnotesize]{javascript}{mStartScreen.usernameText} does not get above 11 chars. 
\begin{minted}[fontsize=\footnotesize]{javascript}
for(var i = 0; i < 15; i++) {
    mStartScreen.keyPressedStart(); // Press Key
    str += strFull.charAt(i);
    CheckSame(mStartScreen.usernameText,str,"testUsernameText" + str);
    global.keyCode++; // go next key
}
\end{minted}
 



\subsection{testDeleteUsername}
\textbf{Author}: Steven Dellamore 
\vspace*{1\baselineskip}
\begin{lstlisting}
async function testDeleteUsername()
\end{lstlisting} 
\vspace*{1\baselineskip}
\textbf{Description}: Does the same thing as testChangeMaxUsername but deletes characters 15 times and checks \mintinline[fontsize=\footnotesize]{javascript}{mStartScreen.usernameText} to ensure that everything has been deleted. \\Note: \mintinline[fontsize=\footnotesize]{javascript}{KeyCode=8} is the delete key. 



\subsection{testCheckSpecialChars}
\textbf{Author}: Steven Dellamore 
\vspace*{1\baselineskip}
\begin{lstlisting}
async function testCheckSpecialChars()
\end{lstlisting} 
\vspace*{1\baselineskip}
\textbf{Description}: Will try to add special chars like ASCII codes \mintinline[fontsize=\footnotesize]{javascript}{10}, \mintinline[fontsize=\footnotesize]{javascript}{240}, \mintinline[fontsize=\footnotesize]{javascript}{33} and then make sure \mintinline[fontsize=\footnotesize]{javascript}{mStartScreen.usernameText} is unchanged because you can't have special chars in ur username. 



\subsection{testHighScoreButton}
\textbf{Author}: Steven Dellamore 
\vspace*{1\baselineskip}
\begin{lstlisting}
async function testHighScoreButton()
\end{lstlisting} 
\vspace*{1\baselineskip}
\textbf{Description}: Sets the mouse positions to be over the high score button. 
\begin{minted}[fontsize=\footnotesize]{javascript}
global.mouseX = mStartScreen.RightX + 1;
global.mouseY = mStartScreen.TopY + 1;
CheckSame(mStartScreen.gameStateStartScreen,0,
    "testCheckInitGameStateScoreButton");
CheckSame(mStartScreen.drawHighScoreButtonCheckMouse(),true,
    "testDrawHighScoreButtonCheckMouse");
\end{minted}
 Then we check that \mintinline[fontsize=\footnotesize]{javascript}{gameStateStartScreen == 0} still equals zero since we havent clicked yet, and check that the high score button is being highlighted correctly. The test then checks if we click on the Score Button \mintinline[fontsize=\footnotesize]{javascript}{gameState == 1}. 



\subsection{testCreateGameButton}
\textbf{Author}: Steven Dellamore 
\vspace*{1\baselineskip}
\begin{lstlisting}
async function testCreateGameButton()
\end{lstlisting} 
\vspace*{1\baselineskip}
\textbf{Description}: Sets the mouse to be over the "Create Game" Button and checks to see if it gets highlighted correctly. Then we click on the button with an empty \mintinline[fontsize=\footnotesize]{javascript}{mStartScreen.usernameText} and check to make sure we did not get moved into the Lobby screen. Finally we add a username \\ \mintinline[fontsize=\footnotesize]{javascript}{mStartScreen.usernameText = "Steven"} and click on the "Create Game" button. We then check we got moved into the lobby screen correctly. 



\subsection{testJoinLobbyButton}
\textbf{Author}: Steven Dellamore 
\vspace*{1\baselineskip}
\begin{lstlisting}
async function testJoinLobbyButton()
\end{lstlisting} 
\vspace*{1\baselineskip}
\textbf{Description}: Sets the \mintinline[fontsize=\footnotesize]{javascript}{mouseX} and \mintinline[fontsize=\footnotesize]{javascript}{mouseY} to be over the "Join Game" button. Then we call \mintinline[fontsize=\footnotesize]{javascript}{mStartScreen.mouseClickedStart()} and check to that we are being put into the token screen correctly. 



\subsection{testCheckLobbyInitValues}
\textbf{Author}: Steven Dellamore 
\vspace*{1\baselineskip}
\begin{lstlisting}
async function testCheckLobbyInitValues()
\end{lstlisting} 
\vspace*{1\baselineskip}
\textbf{Description}: Check the init values when moving to the lobby screen from the start screen. We first check to make sure the \mintinline[fontsize=\footnotesize]{javascript}{Player} object is set corretly like so: 
\begin{minted}[fontsize=\footnotesize]{javascript}
CheckSame(mLobbyScreen.player.username,"Steven","testCheckInitUsername");
CheckSame(mLobbyScreen.player.owner,true,"testCheckInitOwnerTrue");
CheckSame(typeof mLobbyScreen.player.id,"number","testCheckInitID");
\end{minted}
 Then we need to check the \mintinline[fontsize=\footnotesize]{javascript}{Team} object like so: 
\begin{minted}[fontsize=\footnotesize]{javascript}
CheckSame(mLobbyScreen.team.playersInTeam[0].username,"Steven","testCheckInitTeamUsername");
CheckSame(mLobbyScreen.team.playersInTeam[0].owner,true,"testCheckInitTeamOwnerTrue");
CheckSame(typeof mLobbyScreen.team.playersInTeam[0].id,"number","testCheckInitTeamID");
CheckSame(mLobbyScreen.team.teamName,"","testCheckInitTeamName");
CheckSame(typeof mLobbyScreen.team.lobbyToken,"string","testCheckInitLobbyToken");
\end{minted}
 Once these are checked we know that we have good init values. 



\subsection{testCheckTokenIsBeingDisplayed}
\textbf{Author}: Steven Dellamore 
\vspace*{1\baselineskip}
\begin{lstlisting}
async function testCheckTokenIsBeingDisplayed()
\end{lstlisting} 
\vspace*{1\baselineskip}
\textbf{Description}: Checks to see if the Token is being displayed by the frontend in the correct position. This is an example of how we can use Dependency Injection: 
\begin{minted}[fontsize=\footnotesize]{javascript}
var strInside;
var x;
var y;
global.text = function(str, xx, yy) {
    x = xx;
    y = yy;
    strInside = str;
};
mLobbyScreen.drawToken();
CheckSame(strInside,"Token: ","testCheckTextPositionWithNoValue");
CheckSame(x,256,"testCheckYOfTextCall");
CheckSame(y,1454.5454545454545,"testCheckYOfTextCall");
\end{minted}
 As you can see we are checking what \mintinline[fontsize=\footnotesize]{javascript}{drawToken()} is sending the p5 function \mintinline[fontsize=\footnotesize]{javascript}{text()}, which is sent the token, xPos and yPos. 



\subsection{testAddAndRemoveBotsFromLobby}
\textbf{Author}: Steven Dellamore 
\vspace*{1\baselineskip}
\begin{lstlisting}
async function testAddAndRemoveBotsFromLobby()
\end{lstlisting} 
\vspace*{1\baselineskip}
\textbf{Description}: Checks to see if the owner of the lobby can add and remove bots from their lobby. We set \mintinline[fontsize=\footnotesize]{javascript}{mouseX} and \mintinline[fontsize=\footnotesize]{javascript}{mouseY} to the position of the add bot button and then call \mintinline[fontsize=\footnotesize]{javascript}{mouseClickedLobby()} and check if the bot has been increased. 



\subsection{checkPlayCardValues}
\textbf{Author}: Steven Dellamore 
\vspace*{1\baselineskip}
\begin{lstlisting}
async function checkPlayCardValues()
\end{lstlisting} 
\vspace*{1\baselineskip}
\textbf{Description}: Checks the init values of the player cards. Also check that the playcards are being rendered within the bounds of \mintinline[fontsize=\footnotesize]{javascript}{windowWidth} and \mintinline[fontsize=\footnotesize]{javascript}{windowHeight}. 



\subsection{integrationTest1}
\textbf{Author}: Steven Dellamore 
\vspace*{1\baselineskip}
\begin{lstlisting}
async function integrationTest1()
\end{lstlisting} 
\vspace*{1\baselineskip}
\textbf{Description}: This integration test will render the start screen 90,000 times and do different actions at certain times to ensure the start screen as a whole is working correctly. 
\begin{minted}[fontsize=\footnotesize]{javascript}
for(var i = 0; i < 90000; i++) {
    mStartScreen.draw();
    if(i == 500) { 
        // Do Action
    } else if(i == 700) { 
        // Do another action
    } else if(i == 1000) { 
        // Do another action
    } else if(i == 55000) { 
        // Do another action
    }
}
\end{minted}
 Here we are rending the draw method 90,000 times and at different renders we are doing different actions (like mouse clicking, or key pressing). 



\subsection{testGameArrayNotNull}
\textbf{Author}: Richard Hansen 
\vspace*{1\baselineskip}
\begin{lstlisting}
async function testGameArrayNotNull()
\end{lstlisting} 
\vspace*{1\baselineskip}
\textbf{Description}: TODO 



\subsection{testGameScreenRotateKeyPress}
\textbf{Author}: Richard Hansen 
\vspace*{1\baselineskip}
\begin{lstlisting}
async function testGameScreenRotateKeyPress()
\end{lstlisting} 
\vspace*{1\baselineskip}
\textbf{Description}: TODO 



\subsection{testGameScreenFailRotateKeyPress}
\textbf{Author}: Richard Hansen 
\vspace*{1\baselineskip}
\begin{lstlisting}
async function testGameScreenFailRotateKeyPress()
\end{lstlisting} 
\vspace*{1\baselineskip}
\textbf{Description}: TODO 



\subsection{testGameScreenFailRotateKeyPress}
\textbf{Author}: Richard Hansen 
\vspace*{1\baselineskip}
\begin{lstlisting}
async function testGameScreenFailRotateKeyPress()
\end{lstlisting} 
\vspace*{1\baselineskip}
\textbf{Description}: TODO 



\subsection{testFourRotate}
\textbf{Author}: Richard Hansen 
\vspace*{1\baselineskip}
\begin{lstlisting}
async function testFourRotate()
\end{lstlisting} 
\vspace*{1\baselineskip}
\textbf{Description}: TODO 



\subsection{testMove}
\textbf{Author}: Richard Hansen 
\vspace*{1\baselineskip}
\begin{lstlisting}
async function testMove()
\end{lstlisting} 
\vspace*{1\baselineskip}
\textbf{Description}: TODO 



\subsection{testNewSquare}
\textbf{Author}: Richard Hansen 
\vspace*{1\baselineskip}
\begin{lstlisting}
async function testNewSquare()
\end{lstlisting} 
\vspace*{1\baselineskip}
\textbf{Description}: TODO 



\subsection{testNumberOfPlayers}
\textbf{Author}: Richard Hansen 
\vspace*{1\baselineskip}
\begin{lstlisting}
async function testNumberOfPlayers()
\end{lstlisting} 
\vspace*{1\baselineskip}
\textbf{Description}: TODO 



\subsection{testRunnerSetupStartScreen}
\textbf{Author}: Steven Dellamore, Richard Hansen 
\vspace*{1\baselineskip}
\begin{lstlisting}
async function testRunnerSetupStartScreen()
\end{lstlisting} 
\vspace*{1\baselineskip}
\textbf{Description}: TODO 

\end{document}